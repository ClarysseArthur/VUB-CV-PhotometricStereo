\documentclass[nonacm,sigconf,screen]{acmart}

% DO NOT load: hyperref, caption, natbib, titlesec
\usepackage{algorithm}
\usepackage{algpseudocode}
\usepackage{graphicx}
\usepackage{subcaption}
\usepackage{booktabs}
\usepackage{wrapfig}
\usepackage{multicol}
\usepackage{float}
\usepackage{multirow}
\usepackage{amsmath}
\usepackage{amsfonts}
\usepackage{xcolor}
\usepackage{siunitx}
\usepackage{enumitem}
\usepackage{listings}

\newcommand{\cbox}{$\square$}
\newcommand{\cboxchecked}{$\boxtimes$}

\setcounter{secnumdepth}{2}

\citestyle{acmauthoryear}
\bibliographystyle{ACM-Reference-Format}

\acmYear{2025}

\begin{document}

\title{Group Work - Computer Vision}
\subtitle{\textbf{Uncalibrated Photometric Stereo Constrained by Intrinsic Reflectance Image and Shape From Silhoutte}}

\author{Arthur Clarysse}
\email{arthur.clarysse@vub.be}
\affiliation{
  \institution{Vrije Universiteit Brussel}
  \city{Brussels}
  \country{Belgium}
}

\author{Jens Dumortier}
\email{jens.dumortier@vub.be}
\affiliation{
  \institution{Vrije Universiteit Brussel}
  \city{Brussels}
  \country{Belgium}
}

\author{Guillaume Tibergyn}
\email{guillaume.tibergyn@vub.be}
\affiliation{
  \institution{Vrije Universiteit Brussel}
  \city{Brussels}
  \country{Belgium}
}

\pagestyle{plain}
\setcopyright{none}

\setlist[enumerate,1]{label=\arabic*.}
\setlist[enumerate,2]{label=\alph*.}
\setlist[enumerate,3]{label=\roman*.}

\maketitle

\section{Summary}
The reimplementation in this group project was based on the paper "\textit{Uncalibrated photometric stereo constrained by intrinsic reflectance image and shape from silhoutte}" by \citeauthor{hashimoto_uncalibrated_2019}

The main goal of the original paper is to implement an uncalibrated photometric stereo pipeline that estimates surface normals from multiple images without prior knowledge of light directions.
This pipeline first estimates the intrinsic reflectance (or albedo) and then calculates the guide normal \cite{hashimoto_uncalibrated_2019}.

As a dataset, we use the photometric stereo dataset made by Harvard \cite{xiong_shading_nodate}.
The dataset contains multiple shaded images of different animals, including cats, frogs, turtles, and more.
To evaluate our results, we will calculate the average error of the estimated depth map with the actual results provided in the dataset.

\section{Implementation}
\begin{enumerate}
  \item {
    We first started by collecting the data from dataset made by \citeauthor{xiong_shading_nodate}, then, we selected the animal of which we want to make a 3D representation.
    In our case, we chose the cat image, one of the twenty input images can be observed in Figure \ref{fig:implementation:a}.
  }

  \item {
    The second step involved calculating an estimate of the albedo, we did this by calculating the mean of the twenty light angles and applying a bilateral filter on it.
    The result of this step can be seen in Figure \ref{fig:implementation:b}.
  }

  \item {
    In the third step, we calculated different matrices to apply the constant albedo constraint.

    \begin{enumerate}
      \item {
        We started by applying singular value decomposition.
        This produced a matrix $\hat{I}$
      }
      \item{
        From the matrix $\hat{I}$, we could derive $U'$, $W'$, and $V'^{T}$
      }
      \item {
        Finally, we could use these to apply the constant albedo constraint.
      }
    \end{enumerate}
  }

  \item {
    For the fourth step, we calculated the guide normal, the result is visible in \ref{fig:implementation:c}.
    This part of the pipeline is not explained in the paper.
  }

  \item{
    In the final step, we calculated the normal map.
    This could then be used to make a 3D representation of the twenty 2D images.
    The final output of the pipeline is presented in Figure \ref{fig:results}.
  }
\end{enumerate}

\begin{figure}[H]
  \centering
  \begin{subfigure}[t]{0.32\linewidth}
    \centering
    \includegraphics[width=\linewidth]{IMG/cat.png}
    \caption{Example of an image from the dataset, 20 different angles of light exist in the dataset.}
    \label{fig:implementation:a}
  \end{subfigure}\hfill
  \begin{subfigure}[t]{0.32\linewidth}
    \centering
    \includegraphics[width=\linewidth]{IMG/res-01-albedo.png}
    \caption{The estimated albedo of 20 light angles.}
    \label{fig:implementation:b}
  \end{subfigure}\hfill
  \begin{subfigure}[t]{0.32\linewidth}
    \centering
    \includegraphics[width=\linewidth]{IMG/res-02-guidenormal.png}
    \caption{The guide normal}
    \label{fig:implementation:c}
  \end{subfigure}

  \caption{Parts of the pipeline visualized.}
  \label{fig:implementation}
\end{figure}

\begin{figure}[H]
  \centering
  \begin{subfigure}[t]{0.4\linewidth}
    \centering
    \includegraphics[width=\linewidth]{IMG/res-03-normalmap.png}
    \label{fig:results:a}
  \end{subfigure}\hfill
  \begin{subfigure}[t]{0.50\linewidth}
    \centering
    \includegraphics[width=\linewidth]{IMG/res-04-3d.png}
    \label{fig:results:b}
  \end{subfigure}\hfill

  \caption{The 3D results visualized}
  \label{fig:results}
\end{figure}


\section{Results}
\dots

\section{Discussion}
\dots

\section{Reproducibility Checklist}
\begin{itemize}
  \item[\cbox] Code provided \& documented
  \item[\cboxchecked] Data included
  \item[\cboxchecked] Environment included
  \item[\cboxchecked] Run instruction included
  \item[\cboxchecked] Random seeds
  \item[\cbox] Expected outputs
  \item[\cbox] Algorithm pseudo code included
\end{itemize}

\bibliography{sources}

\end{document}