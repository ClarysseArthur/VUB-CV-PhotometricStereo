\documentclass[nonacm,sigconf,screen]{acmart}

% DO NOT load: hyperref, caption, natbib, titlesec
\usepackage{algorithm}
\usepackage{algpseudocode}
\usepackage{graphicx}
\usepackage{subcaption}
\usepackage{booktabs}
\usepackage{wrapfig}
\usepackage{multicol}
\usepackage{float}
\usepackage{multirow}
\usepackage{amsmath}
\usepackage{amsfonts}
\usepackage{xcolor}
\usepackage{siunitx}
\usepackage{enumitem}
\usepackage{listings}

\newcommand{\cbox}{$\square$}
\newcommand{\cboxchecked}{$\boxtimes$}

\setcounter{secnumdepth}{2}

\citestyle{acmauthoryear}
\bibliographystyle{ACM-Reference-Format}

\acmYear{2025}

\begin{document}

\title{Group Work - Computer Vision}
\subtitle{\textbf{Uncalibrated Photometric Stereo Constrained by Intrinsic Reflectance Image and Shape From Silhoutte; A Reimplementation}}

\author{Arthur Clarysse}
\email{arthur.clarysse@vub.be}
\affiliation{
  \institution{Vrije Universiteit Brussel}
  \city{Brussels}
  \country{Belgium}
}

\author{Jens Dumortier}
\email{jens.dumortier@vub.be}
\affiliation{
  \institution{Vrije Universiteit Brussel}
  \city{Brussels}
  \country{Belgium}
}

\author{Guillaume Tibergyn}
\email{guillaume.tibergyn@vub.be}
\affiliation{
  \institution{Vrije Universiteit Brussel}
  \city{Brussels}
  \country{Belgium}
}

\pagestyle{plain}
\setcopyright{none}

\setlist[enumerate,1]{label=\arabic*.}
\setlist[enumerate,2]{label=\alph*.}
\setlist[enumerate,3]{label=\roman*.}

\maketitle

\section{Summary}
The reimplementation in this group project was based on the paper "\textit{Uncalibrated photometric stereo constrained by intrinsic reflectance image and shape from silhoutte}" by \citeauthor{hashimoto_uncalibrated_2019}.
In this paper, they try to estimate the surface normal, which in itself is not a very challenging problem,
However, in our case, the direction of the light sources is unknown, making it much harder to solve.
Mainly because of its ambiguous nature.
Fortunately, it is possible to solve this problem by adding two constraints: the intrinsic reflection (albedo) and an approximate normal.
This results in a reasonable estimate of the surface normal; the estimation and constraints are shown in Figure \ref{fig:overview}.

The main goal of the original paper is to implement an uncalibrated photometric stereo pipeline that estimates surface normals from multiple images without prior knowledge of light directions.
This pipeline first estimates the intrinsic reflectance (or albedo) and the estimated normal, which can be used as constraints.
Finally, using the same matrices, we can calculate the guide normal and use it to make a 3d representation of the image.

As a dataset, we use the photometric stereo dataset made by Harvard \cite{xiong_shading_nodate}.
The dataset contains multiple shaded images of different animals, including cats, frogs, turtles, and more.
To evaluate our results, we will calculate the average error of the estimated depth map with the actual results provided in the dataset.

\begin{figure}[H]
    \centering
    \includegraphics[width=\linewidth]{IMG/Overview.png}
    \caption{Overview of the known and estimated parts of the paper}
    \label{fig:overview}
  \end{figure}

\bigskip
\section{Implementation}
\subsection{Dataset \& Preprocessing}
We first started by collecting the data from dataset made by \citeauthor{xiong_shading_nodate}, then, we selected the animal of which we want to make a 3D representation.
In our case, we chose the cat image, one of the twenty input images can be observed in Figure \ref{fig:implementation:a}.

\subsection{Albedo Estimation}
After the preprocessing, step, we can estimate the albedo.
To do this, we can start by calculating the average image, this results in an image where shades an shadows are reduced.
Unfortunately, some shading effects remain, we can solve this by applying a bilateral filter.

The result from this step is an estimated albedo, and is shown in Figure \ref{fig:implementation:b}.
This result is still unusable in normal photometric stereo techniques, but fortunately, we can use the framework of uncalibrated photometric stereo to solve this.


\subsection{Singular Value Decomposition}
In this step, we want to calculate the true surface matrix and the true light matrix, $S$ and $L$ respectively.

\begin{enumerate}
  \item {
    We start by calculating the shadow matrix $\hat{I}$ (intrinsic illumination matrix) by dividing an input image by the albedo.
    \[
      \hat{i}_{pf} = \frac{i_{pf}}{\hat{a}_p}
    \]
    With
    \begin{itemize}
      \item $i_{pf}$: The pixel $p$ of the image $f$
      \item $\hat{a}_p$: The albedo of image $f$
    \end{itemize}

    This results in the shading image $\hat{I}$
  }
  \item{
    ...
  }
\end{enumerate}


\begin{figure}[H]
  \centering
  \begin{subfigure}[t]{0.32\linewidth}
    \centering
    \includegraphics[width=\linewidth]{IMG/cat.png}
    \caption{Example of an image from the dataset, 20 different angles of light exist in the dataset.}
    \label{fig:implementation:a}
  \end{subfigure}\hfill
  \begin{subfigure}[t]{0.32\linewidth}
    \centering
    \includegraphics[width=\linewidth]{IMG/res-01-albedo.png}
    \caption{The estimated albedo of 20 light angles.}
    \label{fig:implementation:b}
  \end{subfigure}\hfill
  \begin{subfigure}[t]{0.32\linewidth}
    \centering
    \includegraphics[width=\linewidth]{IMG/res-02-guidenormal.png}
    \caption{The guide normal}
    \label{fig:implementation:c}
  \end{subfigure}

  \caption{Parts of the pipeline visualized.}
  \label{fig:implementation}
\end{figure}

\begin{figure}[H]
  \centering
  \begin{subfigure}[t]{0.4\linewidth}
    \centering
    \includegraphics[width=\linewidth]{IMG/res-03-normalmap.png}
    \label{fig:results:a}
  \end{subfigure}\hfill
  \begin{subfigure}[t]{0.50\linewidth}
    \centering
    \includegraphics[width=\linewidth]{IMG/res-04-3d.png}
    \label{fig:results:b}
  \end{subfigure}\hfill

  \caption{The 3D results visualized}
  \label{fig:results}
\end{figure}


\section{Results}
\dots

\section{Discussion}
\dots

\section{Reproducibility Checklist}
\begin{itemize}
  \item[\cbox] Code provided \& documented
  \item[\cboxchecked] Data included
  \item[\cboxchecked] Environment included
  \item[\cboxchecked] Run instruction included
  \item[\cboxchecked] Random seeds
  \item[\cbox] Expected outputs
  \item[\cbox] Algorithm pseudo code included
\end{itemize}

\bibliography{sources}

\end{document}